% LaTeX template adapted from: https://www.overleaf.com/latex/templates/simple-math-homework-template/tbszsswsndrz
\documentclass[landscape,8pt]{extarticle}
\usepackage[utf8]{inputenc}
\usepackage[english]{babel}
\usepackage{extsizes}
\usepackage[]{amsthm} %lets us use \begin{proof}
\usepackage[]{amssymb} %gives us the character \varnothing
\usepackage{amsmath} %for equations
\usepackage[]{listings} %for code blocks
\usepackage{graphicx} %for diagrams
\usepackage{fancyhdr} %for headers
\usepackage[letterpaper, margin=0.25in]{geometry}
\usepackage{tikz} % for drawings
\usepackage{multicol}
\usepackage{ifthen}
\usepackage{enumitem}
\usepackage{listings}
\usepackage{color}

\usepackage{wrapfig}
\usepackage{booktabs}
\setlist{nolistsep}
\setlist[itemize]{leftmargin=0.25pc,itemsep=0em}
\usetikzlibrary{arrows.meta,shapes.arrows,chains,decorations.pathreplacing}
\makeatletter
\renewcommand{\section}{\@startsection{section}{1}{0mm}%
            {-1ex plus -.5ex minus -.2ex}%
            {0.5ex plus .2ex}%x
            {\normalfont\large\bfseries}}
\renewcommand{\subsection}{\@startsection{subsection}{2}{0mm}%
            {-1explus -.5ex minus -.2ex}%
            {0.5ex plus .2ex}%
            {\normalfont\normalsize\bfseries}}
\renewcommand{\subsubsection}{\@startsection{subsubsection}{3}{0mm}%
            {-1ex plus -.5ex minus -.2ex}%
            {1ex plus .2ex}%
            {\normalfont\small\bfseries}}
\makeatother
\definecolor{dkgreen}{rgb}{0,0.6,0}
\definecolor{gray}{rgb}{0.5,0.5,0.5}
\definecolor{mauve}{rgb}{0.58,0,0.82}

\lstdefinelanguage{JavaScript}{
  keywords={typeof, new, true, false, catch, function, return, null, catch, switch, var, if, in, while, do, else, case, break},
  keywordstyle=\color{blue}\bfseries,
  ndkeywords={class, export, boolean, throw, implements, import, this},
  ndkeywordstyle=\color{darkgray}\bfseries,
  identifierstyle=\color{black},
  sensitive=false,
  comment=[l]{//},
  morecomment=[s]{/*}{*/},
  commentstyle=\color{purple}\ttfamily,
  stringstyle=\color{red}\ttfamily,
  morestring=[b]',
  morestring=[b]"
}


\lstset{frame=single,
  language=c,
  aboveskip=3mm,
  belowskip=3mm,
  showstringspaces=false,
  columns=fixed,
  basicstyle={\small\ttfamily},
  numbers=none,
  breaklines=true,
  breakatwhitespace=true,
  tabsize=3
}
\newcommand{\code}{\lstinline}
\graphicspath{}
\pagestyle{empty}
\ifthenelse{\lengthtest { \paperwidth = 11in}}
{ \geometry{top=.5in,left=.5in,right=.5in,bottom=.5in} }
{\ifthenelse{ \lengthtest{ \paperwidth = 297mm}}
{\geometry{top=1cm,left=1cm,right=1cm,bottom=1cm} }
{\geometry{top=1cm,left=1cm,right=1cm,bottom=1cm} }
}
\setlength{\parindent}{0em}
\setlength{\parskip}{-0.25em}
\setlength{\headheight}{16pt}
\pagestyle{fancy}
\fancyhf{}
\renewcommand{\headrulewidth}{0pt}
\rhead{Rob Phillips | CruzID: rogphill | Student ID: 1597347}

\begin{document}
    \footnotesize
    \begin{multicols}{4}
    \setlength{\premulticols}{1pt}
    \setlength{\postmulticols}{1pt}
    \setlength{\multicolsep}{1pt}
    \setlength{\columnsep}{2pt}
    \begin{itemize}
    \item \code{Tuples and Relations:}
    \begin{itemize}
        \item \emph{Tuple:} a $k$\emph{-tuple} is an ordered sequence of $k$ values
        \item If $D_1, D_2, \ldots, D_k$ are sets of elements then the cartesian product $D_1 \times
        D_2 \times \ldots D_k$ is the set of all $k$-tuples $(d_1, d_2, \ldots, d_k)$ such that
        $\forall 1 \leq i \leq k:\ d_i \in D_i$
        \item \emph{Relation:}
        \begin{itemize}
            \item A $k$\emph{-ary relation} is a subset of $D_1 \times D_2 \times \ldots D_k$ where
            each $D_i$ is a set of elements
            \item $D_i$ is the \emph{domain (or datatype)} of the $i^{th}$ column of the relation
            \item Domains may be enumerated $\{``AMS'', ``CMPS'', ``TIM''\}$ or may be of standard
            types 
        \end{itemize}
        \item An \emph{attribute} is the name of a column in a relation
        \item A \emph{relation schema} $R$ is a set $\{A_1, \ldots, A_k\}$ of attributes written
        $R(A_1, \ldots, A_k)$, where $A_i$ is the name of the $i^{th}$ column. 
        \item A \emph{relation database schema} or \emph{database schema} is a set of 
        schemas with disjoint relation names.
    \end{itemize}
    \item \code{SQL Primitives:}
    \begin{itemize}
        \item \code{CHAR(n)}: fixed-length string of up to $n$ characters (blank-padded with
        trailing spaces)
        \item \code{VARCHAR(n)}: also a string of up to $n$ characters
        \item \code{BIT(n)}: padded on the right with \code{0}s.
        \item \code{BIG VARYING(n)}: works like \code{VARCHAR}
        \item \code{BOOLEAN}: true, false, unknown
        \item \code{INT} or \code{INTEGER}: works like in C
        \item \code{SHORTINT}: works like \code{short int}
        \item \code{DECIMAL(n, d)}, \code{NUMERIC(n, d)}: total of $n$ digits, $d$ of them to the
        right of the decimal point
        \item \code{FLOAT(p)}, \code{FLOAT}, \code{REAL}
        \item \code{DOUBLE PRECISION}: analagous to \code{double} in c
        \item \code{DATE}, \code{TIME}, \code{TIMESTAMP}, \code{INTERVAL}: constants are character
        strings of specific form e.g. \code{DATE `2017-09-13'}
        \begin{itemize}
            \item \code{TIME A - TIME B} results in an \code{INTERVAL}
            \item \code{TIME A + INTERVAL B} results in a \code{TIME}
            \item Similarly for \code{TIMESTAMP} and \code{DATE}
        \end{itemize}
    \end{itemize}
    \item \code{Tables:} 
    \begin{itemize}
    \item A \emph{key} of a relation schema $R$ is a subset $K$ of the attributes of $R$ such that:
        \begin{itemize}
            \item $\forall Y \in R: K \rightarrow Y$
            \item No proper subset of $K$ has the above property
        \end{itemize}
        \item A \emph{superkey} is a set of attributes that includes a key
        \item \code{CREATE TABLE R(A, B, C, PRIMARY KEY(A))}:
        \begin{itemize}
        \item None of the tuples in \code{R} can have null \code{A} values
        \item Rows are uniquely identified by their \code{A} values
        \item There can be $\leq 1$ primary key for a table
        \end{itemize}
        \item \code{CREATE TABLE S(D, E, F, UNIQUE(D))}:
        \begin{itemize}
            \item Rows in \code{S} can contain null \code{D} values
            \item Rows with \emph{non-null} \code{D} values are uniquely identified by their \code{D} values
            \item There can be multiple unique constraints in addition to a primary key
        \end{itemize}
        \item \code{CREATE TABLE T(G NOT NULL, H DEFAULT `foo')}:
        \begin{itemize}
            \item If no default value is specified and no value is entered then the value will be \code{NULL}
            \item \code{NOT NULL} prevents null values
        \end{itemize}
    \end{itemize}
    \item \code{Queries:}
    \begin{itemize}
        \item Basic form: \\
\code{SELECT [DISTINCT] c1, c2, ..., cm} \\
\code{FROM R1, R2, ..., Rn} \\
\code{[WHERE condition]}
    \item \code{SELECT}:
    \begin{itemize}
        \item Projection: \code{SELECT title, year}
        \item \code{DISTINCT}: Removes duplicate tuples from result
        \item Aliasing: \code{SELECT title AS name}
        \item Expressions are allowed in the \code{SELECT} clause. Ex: \code{SELECT title AS name, length * 60 AS durationInSeconds}
        \item Constants can also be included: \code{SELECT title AS name, length * 60 AS durationInSeconds, `seconds' AS inSeconds}
    \end{itemize}
        \item \code{WHERE}:
        \begin{itemize}
            \item Comparison operators: \code{=}, \code{<>}, \code{<}, \code{>}, \code{<=}, \code{>=}
            \item Logical connectives: \code{AND}, \code{OR}, \code{NOT}
            \item Arithmetic expressions: \code{+}, \code{-}, \code{*}, \code{/}, etc
            \item In general the \code{WHERE} clause is a boolean expression where each condition is
            of the form \emph{expression op expression}
        \end{itemize}
        \item Pattern matching with the \code{LIKE} operator:
        \begin{itemize}
            \item $s$ \code{LIKE} $p$, $s$ \code{NOT LIKE} $p$
            \item $s$ is a string, $p$ is a pattern
            \item `\code{\%}' stands for \code{0} or more arbitrary characters
            \item `\code{_}' stands for exactly one arbitrary character
            \item quotes: \code{WHERE x LIKE ''''} matches one 
            \item quotes: \code{WHERE x LIKE ''''''} matches two 
            \item  \code{\%} or \code{_}: \code{WHERE x LIKE `!\%\%!' ESCAPE '!'} where
            \code{!} can be any character
        \end{itemize}
        \item \code{DATE} and \code{TIME} and \code{TIMESTAMP}
        \begin{itemize}
            \item Separate data types
            \item Constants are character strings of the form: \\
\code{DATE '2015-01-13'} \\
\code{TIME '16:45:33'} \\
\code{TIMESTAMP '2015-01-13 16:45:33'} 
            \item \code{DATE}, \code{TIME}, \code{TIMESTAMP} can be compared using ordinary
            comparison operators e.g. \code{WHERE ReleaseDate <= DATE '1990-06-19'}
\end{itemize}
        \item If \code{Salary} is \code{NULL} then the following will be \code{UNKNOWN}:
        \begin{itemize}
            \item \code{Salary = 10}
            \item \code{Salary <> 10}
            \item \code{90 > Salary OR 90 <= Salary}
            \item \code{Salary = NULL}
            \item \code{Salary <> NULL}
            \end{itemize}
        \item Ordering the result:
        \begin{itemize}
            \item \code{ORDER BY} presents the result in a sorted order
            \item By default the result will be ordered in ascending order \code{ASC}
            \item For descending order on an attribute you write \code{DESC} in the list of attributes
        \end{itemize}
        \item Multiple relations in \code{FROM} clause: for every tuple $t_1 \in R_1, t_2 \in R_2,
        \ldots, t_n from R_n$ if $t_1, \ldots, t_n$ satisfy \emph{condition} then add the resulting
        tuple that consists of $c_1, c_2, \ldots, c_m$ components of $t$ into the result
    \end{itemize}
    \item \code{Joins:} With relations \code{R(A,B,C)} and \code{S(C,D,E)}
    \begin{itemize}
        \item \code{R JOIN S ON R.B=S.D AND R.A=S.E}:
        \begin{itemize}
            \item Selects only tuples from \code{R} and \code{S} where \code{R.B=S.D} and \code{R.A=S.E}
            \item Schema of the resulting relation: \code{(R.A, R.B, R.C, S.C, S.D, S.E)}
            \item Equivalent to: \\
            \code{SELECT *} \\
            \code{FROM R, S} \\
            \code{WHERE R.B=S.D AND R.A=S.E;}
        \end{itemize}
        \item \code{R CROSS JOIN S}:
        \begin{itemize}
            \item Product of the two relations \code{R} and \code{S}
            \item Schema of the resulting relation: \code{(R.A, R.B, R.C, S.C, S.D, S.E)}
            \item Equivalent to: \\
            \code{SELECT *} \\
            \code{FROM R, S;}
        \end{itemize}
        \item \code{R NATURAL JOIN S}:
        \begin{itemize}
            \item Schema of the resulting relation: \code{(A, B, C, D, E)}
            \item Equivalent to: \\
            \code{SELECT R.A, R.B, R.C, S.D, S.E} \\
            \code{FROM R, S} \\
            \code{WHERE R.C = S.C}
        \end{itemize}
    \end{itemize}
    \item \code{Set and Bag Operations:} \code{R(A,B,C), S(A,B,C)}
    \begin{itemize}
        \item \code{UNION}: Set union
        \begin{itemize}
            \item Input to union must be \emph{union-compatible}: \code{R} and \code{S} must have
            attributes of the same type, in the same order
            \item Output of the union has the same schema as \code{R} or \code{S}
            \item Meaning: Output consists of the \emph{set} of all tuples from \code{R} and from \code{S}
            \item Could have been called \code{UNION DISTINCT} \\
            \code{(SELECT * FROM R)} \\
            \code{UNION} \\
            \code{(SELECT * FROM S)} 
        \end{itemize}
        \item \code{UNION ALL}: Bag union
        \begin{itemize}
            \item Input must be \emph{union-compatible}
            \item Output has the same schema as \code{R} or \code{S}
            \item Output consists of the collection of all tuples from \code{R} and from \code{S} \emph{including duplicates}.
            \item Attributes/column names may be different - \code{R}'s are used
        \end{itemize}
        \item \code{INTERSECT}, \code{INTERSECT ALL}: set/bag intersection
        \begin{itemize}
            \item Input must be union-compatible.
            \item $Query_1$ \code{INTERSECT} $Query_2$ 
            \item $Query_1$ \code{INTERSECT ALL} $Query_2$
            \item Find all tuples that are in the results of both $Query_1$ and $Query_2$.
            \item \code{INTERSECT} is distinct. \code{INTERSECT ALL} reports duplicates.
        \end{itemize}
        \item \code{EXCEPT}, \code{EXCEPT ALL}: set difference, bag difference
        \begin{itemize}
            \item Must be union-compatible
            \item $Query_1$ \code{EXCEPT} $Query_2$
            \item $Query_1$ \code{EXCEPT ALL} $Query_2$
            \item Find all tuples that are in the result of $Query_1$ and not in the result of $Query_2$
            \item \code{EXCEPT} is distinct, \code{EXCEPT ALL} is not
        \end{itemize}
        \item Order of operations: \code{INTERSECT} has higher precedence than \code{UNION} and \code{EXCEPT}.
    \end{itemize}
    \item \code{Subqueries:}
    \begin{itemize}
        \item A query embedded in another query
        \item Can be used as a boolean or can return a constant or can return a relation
        \item \code{IN}, \code{NOT IN}: used to select from subquery that returns relation
        \item \code{WHERE} $A$ \code{< ANY}: checks that attribute $A$ is less than at least one of
        the answers returned by the subquery.
        \item \code{EXISTS}: Checks that subquery returns non-empty result. Also: \code{NOT EXISTS}
    \end{itemize}
    \item \code{Aggregates and Grouping:}
    \begin{itemize}
    \item Basic SQL has 5 aggregation operators: \code{SUM}, \code{AVG}, \code{MIN}, \code{MAX}, \code{COUNT}
    \item Aggregation operators work on scalar values, except for \code{COUNT(*)} which counts the
    number of tuples
    \item \code{GROUP BY} clause follows the \code{WHERE} clause
    \begin{itemize}
        \item Let Result begin as an empty multiset of tuples
        \item For every tuple $t_1$ from $R_1$, $t_2$ from $R_2, \ldots, t_n$ from $R_n$: if $t_1,
        \ldots, t_n$ satisfy \emph{condition} then add the resulting tuple that consist of $c_1,
        c_2, \ldots, c_m$ of the $t_i$ into Result
        \item Group the tuples according to the grouping attributes - if \code{GROUP BY} is omitted,
        the entire table is one group
    \end{itemize}
    \item \code{NULL}s are ignored in any aggregation
    \begin{itemize}
        \item They do not contribute to the \code{SUM}, \code{AVG}, \code{COUNT}, \code{MIN},
        \code{MAX} of an attribute
        \item \code{COUNT(*)} = the number of tuples in a relation even if some columns are \code{NULL}
        \item \code{COUNT(A)} is the number of tuples with \emph{non-}\code{NULL} values for $A$
        \item \code{SUM}, \code{AVG}, \code{MIN}, \code{MAX} on an empty result (no tuples) is \code{NULL}
        \item \code{COUNT} of an empty result is \code{0}
        \item \code{GROUP BY} does \emph{not} ignore \code{NULL}
    \end{itemize}
    \item \code{HAVING} clause:
    \begin{itemize}
        \item Choose groups based on some aggregate property of the group itself
        \item Same attributes and aggregates that can appear in the \code{SELECT} can appear in the
        \code{HAVING} clause condition
        \item Can use \code{EVERY} to constrain \code{HAVING} to all tuples in the group e.g. \code{HAVING COUNT(*) > 1 AND EVERY (S.age <= 40)}
    \end{itemize}
\end{itemize}
\item \code{Database Modification Statements:}
\begin{itemize}
    \item \code{INSERT INTO R(A1, ..., An) VALUES (v1, ..., vn)}: a tuple $(v_1, \ldots, v_n)$ is
    inserted into $R$ such that $A_i = v_i \forall i$ and default values (perhaps \code{NULL}) are
    entered for any missing attributes.
    \item \code{DELETE FROM R WHERE <condition>}: Deletes \emph{all} tuples such that the condition
    evaluates as true - if there is no \code{WHERE} clause it will delete all tuples in $R$
    \item \code{UPDATE R SET <new-value-assignments> WHERE <condition>}: Change the given attribute
    to the new value in every tuple in $R$ where the condition is true
    \item Semantics: database modifications are completely evaluated on the old state of the
    database producing a new state of the database
\end{itemize}
\item \code{Transaction:}
\begin{itemize}
    \item Transactions provide ACID properties: atomicity, consistency, isolation, durability
    \item \code{START TRANSACTION} or \code{BEGIN TRANSACTION}: marks the beginning of a
    transaction, followed by one or more SQL statements
    \item \code{COMMIT}: Ends the transaction. All changes are durably written to the backing store
    and become visible to other transactions.
    \item \code{ROLLBACK}: Causes the transaction to abort or terminate. None of the changes are committed.
    \item \code{SET TRANSACTION READ ONLY}: 
    \begin{itemize}
        \item set \emph{before} the transaction begins, tells the SQL system that the next
        transaction is read-only
        \item SQL uses this to parallelize many read-only transactions
    \end{itemize}
    \item \code{SET TRANSACTION READ WRITE}:
    \begin{itemize}
        \item Tells SQL that the next transaction may write data in addition to read
        \item Default option if not specified, often not specified
    \end{itemize}
    \item Dirty Reads: \emph{Dirty data} refers to data that is written by a transaction but has not
    yet been committed by the transaction
    \item Isolation levels:
    \begin{itemize}
        \item \code{SET TRANSACTION READ WRITE ISOLATION LEVEL READ UNCOMMITTED}
        \item Default isolation level depends on system, most run with \code{READ COMMITTED} or \code{SNAPSHOT ISOLATION}
        \item \code{READ COMMITTED}: only clean(committed) reads but you might read data committed
        by other transactions
        \item \code{REPEATABLE READ}: repeated queries of a tuple during a transaction will retrieve
        the same value. Also, a second scan may return `phantoms' which are tuples newly inserted
        while the transaction is running.
        \item \code{SERIALIZABLE}: Can be replayed one by one.
    \end{itemize}
\end{itemize}
    \item \code{Constraints:}
    \begin{itemize}
        \item \code{Key/Unique} constraint: No repetitions of this value in the relation
        \item \code{Foreign-key} constraint: Referential integrity. Value must exist in another
        relation as specified. Referenced attributes must be \code{PRIMARY KEY} or \code{UNIQUE}.
        \begin{itemize}
            \item With attribute:\\
            \code{beer CHAR(20) REFERENCES Beers(name)}
            \item As schema element:\\
        \code{FOREIGN KEY (beer)}\\
        \code{  REFERENCES Beers(name)}
        \end{itemize}
        \item \code{Value-based} constraint: \code{CHECK(<condition>)}. Checked on insert or update.
        \item Enforcing constraints:
        \begin{itemize}
            \item \emph{Default:} Reject the modification
            \item \emph{Cascade:} Make changes to maintain consistency
            \item \emph{Set NULL:} Change dependent values to \code{NULL}
            \item Selected independently with \code{ON [UPDATE, DELETE] [SET NULL, CASCADE]}
        \end{itemize}
        \item \code{Assertions:} database-schema elements, like relations or views. Defined by:\\
        \code{CREATE ASSERTION <name>} \\
        \code{  CHECK (<condition>)}
    \end{itemize}
    \item \code{Triggers:}
    \begin{itemize}
        \item Event-Condition-Action or \emph{ECA} rule:
        \begin{itemize}
            \item \emph{Event:} Typically a DB modification
            \item \emph{Condition:} Any SQL boolean expression
            \item \emph{Action:} Any SQL statements
        \end{itemize}
        \item \emph{Create:}
        \begin{itemize}
            \item \code{CREATE TRIGGER <name>}
            \item \code{CREATE OR REPLACE TRIGGER <name>}
        \end{itemize}
        \item \emph{The Event:}
        \begin{itemize}
            \item \code{[AFTER, BEFORE] [INSERT, DELETE, UPDATE [ON]]}
            \item Can be \code{INSTEAD OF} if the relation is a view
        \end{itemize}
        \item \code{FOR EACH ROW}
        \begin{itemize}
            \item Triggers are `row-level' or `statement-level'
            \item \code{FOR EACH ROW} means row-level, absence means statement-level
            \item \emph{Row level triggers} execute once for each tuple
            \item \emph{Statement-level} execute once for a SQL statement
        \end{itemize}
        \item \code{REFERENCING}
        \begin{itemize}
            \item \code{INSERT} statements imply a new tuple or table
            \item \code{DELETE} implies an old tuple or table
            \item \code{UPDATE} implies both
            \item \code{[NEW OLD] [TUPLE, TABLE] AS <name>}
        \end{itemize}
        \item \emph{The Condition:}
        \begin{itemize}
            \item Any boolean condition
            \item Evaluated on the DB as it existed \code{BEFORE}/\code{AFTER} the event
        \end{itemize}
        \item \emph{The Action:}
        \begin{itemize}
            \item There can be more than one SQL statement
            \item Surround with \code{BEGIN ... END}
        \end{itemize}
    \end{itemize}
    \item \code{Relational Algebra:}
    \begin{itemize}
        \item Selection: $\sigma_{condition}(R)$
        \begin{itemize}
            \item \emph{Input:} Relation with schema $R(A_1, \dots, A_n)$
            \item \emph{Output:} Relation with attributes $A_1, \dots, A_n$
            \item \emph{Meaning:} Extracts rows which satisfy condition
        \end{itemize}
        \item Projection: $\pi_{<attribute\ list>}(R)$
        \begin{itemize}
            \item \emph{Input:} Relation with schema $R(A_1, \dots, A_n)$
            \item \emph{Output:} Relation with listed attributes
            \item \emph{Meaning:} Extracts all rows from $R$ and outputs only those attributes listed
        \end{itemize}
            \item Union: $R \cup S$
            \begin{itemize}
                \item \emph{Input:} Union-compatible relations $R,S$
                \item \emph{Output:} Relation with the same type of $R$ (or $S$)
                \item \emph{Meaning:} The output is the set of all tuples in either $R$ or $S$ or
                both
                \item Both \emph{commutative} and \emph{associative}
            \end{itemize}
            \item Set difference: $R - S$
            \begin{itemize}
                \item \emph{Input:} Union-compatible relations $R,S$
                \item \emph{Output:} Relation with the same type of $R$ (or $S$)
                \item \emph{Meaning:} $\{x\ | x \in R,\ x \notin S\}$
            \end{itemize}
            \item Cross-product: $R \times S$
            \begin{itemize}
                \item \emph{Input:} $R(A_1, \dots, A_n), S(B_1, \dots, B_m)$
                \item \emph{Output:} $T(A_1, \dots, A_n, B_1, \dots, B_m)$
                \item \emph{Meaning:} $R \times S =$ $ \{(a_1, \dots, a_n, b_1, \dots, b_m) $ $| (a_1,
                \dots, a_m) \in R,\ (b_1, \dots, b_m) \in S) \}$
            \end{itemize}
            \item Intersection: $R \cap S$ is a \emph{derived operator} \\
            $R \cap S = R - (R - S) = S - (S - R)$
        \item Renaming: $\rho_{S(A_1, \dots, A_n)}(R)$
        \begin{itemize}
            \item \emph{Input:} A relation $R$ and a set of attributes $\{B_1, \dots, B_n\}$
            \item \emph{Output:} A relation $S$ and attributes $A_1, \dots, A_n$
        \end{itemize}
        \item Natural Join: $R \bowtie S$
        \begin{itemize}
            \item \emph{Input:} Two relations $R$ and $S$ where $\{A_1, \dots, A_k\}$ is a set of
            common attributes between them
            \item \emph{Output:} A relation where its attributes are $attr(R) \cup attr(S)$.
            Consists of $R \times S$ without any repeats of the common attributes.
            \item \emph{Meaning:} $R \bowtie S = \pi_{(attr(R) \cup attr(S))} (\sigma_{C} (R \times
            S))$ (where $C = R.A_1 = S.A_1 \text{ AND } \dots \text{ AND } R.A_k = S.A_k$)
        \end{itemize}
        \item Semi-Join: $R \ltimes S = \pi_{attr(R)}(R \bowtie S)$
        \item Theta Join: $\bowtie_\theta$
        \begin{itemize}
            \item \emph{Input:} $R(A_1, \dots, A_n), S(B_1, \dots, B_m)$
            \item \emph{Output:} $T(A_1, \dots, A_n, B_1, \dots, B_m)$ Identical attributes are
            disambiguated with the relation names.
            \item \emph{Meaning:} Equivalent to writing $\sigma_{\theta}(R \times S)$.
        \end{itemize}
        \item Division: $R/S$ or $R \div S$
        \begin{itemize}
            \item \emph{Input:} Two relations $R$ and $S$ such that $attr(S) \subset attr(R)$ and
            $attr(S) \neq \emptyset$
            \item \emph{Output:} Relation whose attributes are in $attr(R) - attr(S)$
            \item \emph{Meaning:} Given $R(a,b,c,d)$, $S(b,d)$: $R \div S$ outputs $(a,c)$ for each
            tuple in $S$ such that $R.b = S.b$ and $R.d = S.d$
        \end{itemize}
        \item \emph{Independence:} The five basic operators are indepdent of each other.
        \begin{itemize}
            \item $\times$ increases columns
            \item $\cup$ increases rows
            \item $\pi$ decreases columns
            \item $\sigma$ is binary, decreases rows
            \item $-$ is unary, decreases rows
        \end{itemize}
    \end{itemize}
    \item \code{STORED PROCEDURES}
    \begin{itemize}
        \item \code{PSM} or \emph{persistent stored modules} store procedures as DB schema elements
        \item Basic form:\\
\code{CREATE PROCEDURE <name> (}\\
\code{  <parameter list>) RETURNS <type>}\\
\code{  <optional local declarations>}\\
\code{  <body>;}
        \item Uses mode-name-type triples where the mode can be:
        \begin{itemize}
            \item \code{IN}: procedure uses value, doesn't change 
            \item \code{OUT}: procedure changes value, doesn't use
            \item \code{INOUT}: both
        \end{itemize}
        \item Function parameters must be of type \code{IN}
        \item Procedures invoked with \code{CALL <name>(<parameters>)}
        \item May be used in SQL expressions wherever their return type fits
        \item \code{RETURN <expression>} sets the return value but doesn't terminate execution
        \item \code{DECLARE <name> <type>} declares local variable
        \item \code{BEGIN ... END} for groups of statements
        \item \code{SET <variable> = <expression>}
        \item \code{IF <condition> THEN <statements> END IF;}
        \item Add \code{ELSE <statement(s)>} as \code{IF...THEN...ELSE...END IF}
        \item Add additional cases: \code{IF...THEN...ELSEIF...ELSE}
        \item Basic loop:\\
\code{<loop name>: LOOP}\\
\code{<statements>}\\
\code{END LOOP;}
        \item Leave loop with \code{LEAVE <loop name>}
        \item Also: \code{WHILE <condition> DO <statements> END WHILE;}
        \item Also: \code{REPEAT <statements> UNTIL <condition> END REPEAT;}
        \item Queries producing one value can be the expression in an assignment.
        \item Queries returning one row: \code{SELECT ... INTO ...}
        \item Cursors:
        \begin{itemize}
            \item \code{DECLARE c CURSOR FOR <query>} to declare, binds values
            \item \code{OPEN c} to open
            \item \code{CLOSE c} to close
            \item \code{FETCH FROM c INTO x1, x2, ..., xn} sets the $x$'s to the values of a tuple
            \item \code{c} moves to the next tuple automatically
            \item \code{DECLARE NotFound CONDITION FOR SQLSTATE '02000'}
            \item \code{CURRENT OF c} allows use in \code{WHERE} for current tuple
        \end{itemize}
    \end{itemize}
    \item \code{EMBEDDED SQL}
    \begin{itemize}
        \item Uses preprocessor in host language to turn SQL statements into host library calls
        \item Declare shared variables with \\
\code{EXEC SQL BEGIN DECLARE SECTION;}\\
\code{<host-language declarations>}\\
\code{EXEC SQL END DECLARE SECTION;}
        \item Shared variables must be preceded by a colon. Can be used as if they were constants
        provided by the host language program. Can get values from SQL statements and pass those
        values to the host language program. In the host language they behave like any other
        variable.
        \item Usage: \code{<SQL varname> = :<host varname>}
        \item Every SQL statement must begin \code{EXEC SQL ...}
        \item PrepareStatement:\\
\code{EXEC SQL PREPARE <query-name>}\\
\code{  FROM <text of query>;}
        \item Execute: \code{EXEC SQL EXECUTE <query-name>;}
        \item If only using it once:
\code{EXEC SQL EXECUTE IMMEDIATE <text>;}
        \end{itemize}
        \item \code{JDBC}
    \begin{itemize}
        \item Provides \code{Statement} and \code{PreparedStatement} classes
        \item Usage:\\
\code{import java.sql.*;}\\
\code{Class.forName(com.mysql.jdbc.Driver);}\\
\code{Connection myCon = DriverManager.getConnection(...);}
        \item \code{createStatement()} takes a string of a SQL statement
        \item \code{prepareStatement()} is configurable
        \item Use \code{executeQuery} or \code{executeUpdate} as appropriate
        \item Both return \code{ResultSet} which uses something like a cursor, \code{rs.next()} gets
        the next one
        \item When a resultSet returns a tuple we get the components with \code{getX(i)} where
        \code{X} is a type and \code{i} is the component number
    \end{itemize}
    \item \code{FUNCTIONAL DEPENDENCIES}
    \begin{itemize}
        \item Let $R$ be a relation schema. An FD is an integrity constraint $X \rightarrow Y$ where
        $X$ and $Y$ are non-empty subsets of the attributes of $R$
        \item A relation instance $r$ of $R$ \emph{satisfies} the FD $X \rightarrow Y$ iff for every
        pair of tuples $t, t^\prime \in r: t[X] = t^\prime[X] \implies t[Y] = t^\prime[Y]$
        \item A set $\mathcal{F}$ of FDs implies an FD $F$ if for every instance $r$ that satisfies
        $\mathcal{F}$, it must be true that $r$ satisfies $F$: $\mathcal{F} \models F$
        \item \code{ARMSTRONG'S AXIOMS}
        \begin{itemize}
            \item \emph{Reflexivity:} $Y \subseteq X \implies X \rightarrow Y$ (trivial)
            \item \emph{Augmentation:} $X \rightarrow Y \implies \forall Z:\ XZ \rightarrow YZ$
            \item \emph{Transitivity:} $X \rightarrow Y,\ Y \rightarrow Z \implies X \rightarrow Z$
        \end{itemize}
        \item $\mathcal{F} \vdash F$ means $F$ can be derived from $\mathcal{F}$ using Armstrong's
        Axoms ($\mathcal{F}$ generates $F$)
        \item Closure: $\mathcal{F}^+$ means the set of all FDs implied by $\mathcal{F}$
        \item $\mathcal{F} \vdash X \rightarrow A \iff A \in X^+$
    \end{itemize}
    \item \code{NORMAL FORMS}
    \begin{itemize}
        \item \emph{First Normal Form (1NF):} Very basic requirement of the data model. Type of
        every attribute must be atomic.
        \item \emph{Boyce-Codd Normal Form (BCNF):} Let $R$ be a relation schema, $\mathcal{F}$ a
        set of FDs that holds for $R$, $A$ an attribute in $R$, and $X$ a subset of attributes in
        $R$. $R$ is in $BCNF$ iff for every FD $X \rightarrow A$: $A \in X$, or $X$ is a superkey.
        \item \emph{Third Normal Form (3NF):} Like $BCNF$ except $A$ can be a part of a key of $R$.
    \end{itemize}
    \item \code{DECOMPOSITION}
    \begin{itemize}
        \item Given a relation $R$, a decomposition is a set of attributes $X_1, \dots, X_k$ (not
        necessarily disjoint) such that each $X_i \subseteq attr(R)$ and $X_1 \cup \dots \cup X_k =
        attr(R)$
        \item $\pi_{X_i}(R) = R_i$ with instances of $R$ written $r$ and instances of $R_i$ written $r_i$
        \item Used to eliminate anomalies
        \item \emph{Lossless Join Decomposition} if it allows recreation of $R$ by using $r_1
        \bowtie \dots \bowtie r_k$
        \item Lossless decomposition of $R$ into $R_1(X_1), R_2(X_2)$ requires $\mathcal{F}^+$ has
        either $X_1 \cap X_2 \rightarrow X_1$ or $X_1 \cap X_2 \rightarrow X_2$.
        \item The FDs on $R_i$ are the FDs in $\mathcal{F}^+$ that mention only $attr(R)$
        \item The decomposition is \emph{dependency-preserving} if, when the $R_i$'s are rejoined,
        the FDs that were on the $R_i$s imply all of the original FDs in $\mathcal{F}$
        \item Given a schema and set of FDs, it's always possible to decompose into a set of 
        \begin{itemize}
            \item $BCNF$ relations which eliminate anomalies and is a lossless join decomposition
            but might not be dependency-preserving
            \item $3NF$ relations which is a lossless join decomposition and is
            dependency-preserving but might not always eliminate anomalies
        \end{itemize}
    \end{itemize}
    \item \code{SEMI-STRUCTURED DATA}
    \begin{itemize}
        \item \code{XML}
        \begin{itemize}
            \item \emph{Tags} come in pairs \code{<data>...</date>}
            \item Must be properly nested
            \item Start the doc with \code{<?xml ... ?>}
            \item Normal declaration is \code{<?xml version = "1.0" standalone = "yes" ?>}
            \item An attribute is anything surrounded by a tag
            \item A list is multiple tags used repeatedly
            \item A tag attribute goes inside the tag e.g. \code{<word lang="en">}
            \item Can use \code{id="foo"} attribute to specify an element and then reference it with
            \code{idrefs="foo"}
            \item DTD specifies syntax for a document. DTD structure:\\
\code{<!DOCTYPE <root tag> [}\\
\code{  <!ELEMENT <name>(<components>)>}\\
\code{]>}
            \item Uses regex \code{?*()|+,}
            \item \code{(#PCDATA)} means \emph{parsed character data} e.g. text
            \item \code{#IMPLIED} means optional, otherwise \code{#REQUIRED}
            \item Use \code{ID} and \code{IDREF(S)} in DTD to enforce constraints
            \item Use \code{SYSTEM "foo"} in doctype tag to specify source for dtd
        \end{itemize}
        \item \code{JSON}
        \begin{itemize}
            \item Unordered sets of name/value pairs
            \item Normal: \lstinline|{a:b, c:d}|
            \item Arrays are ordered: \code{[a,b,c,]}
            \item Strings wrapped in double quotes support backslash escapes
            \item Numbers: integer, real, scientific
        \end{itemize}
    \end{itemize}
    \item \code{STAR SCHEMA}
    \begin{itemize}
        \item Common organization for OLAP data in a warehouse
        \item \code{DIMENSION TABLE}
        \begin{itemize}
            \item Smaller, largely static info describing the data that's in the Fact
            \item 
        \end{itemize}
        \item \code{FACT TABLE}
        \begin{itemize}
            \item Large accumulation of facts such as sales
            \item The key consists of values from dimension tables (foreign keys)
            \item Facts usually `insert mostly' with rare updates
            \item \emph{Dimension Attributes:} The key of a dimension table, references Dimenson Tables
            \item \emph{Dependent Attributes:} A fact value determined by the dimension attributes
        \end{itemize}
        \item \emph{Roll-up} means aggregate along one or more dimensions
        \item \emph{Drill-Down} means `dis-aggregate' or break an aggregate into its constituent parts
    \item \code{R [LEFT, RIGHT, FULL] OUTER JOIN S [ON <condition>]}: 
    \begin{itemize}
        \item Used to join $R$ and $S$ where there exists some tuple in $R$ which has no match in
        $S$
        \item Fills in missing attributes wiht \code{NULL}
        \item \code{LEFT} means pad dangling tuples of $R$ only (and vice versa)
        \item \code{FULL} means pad both, default
    \end{itemize}
    \item Taking the cartesian product of the dimension table keys, then taking 
    \code{LEFT OUTER JOIN} of that with the fact table, will give you entries for every combination of dimensions
    (not just those that have entries in the fact table)
    \item \code{COALESCE(x, 0)} has value $x$ if $x$ isn't \code{NULL}, otherwise \code{0}.
    \end{itemize}
    \item \code{ACID}
    \begin{itemize}
        \item \code{ATOMICITY}: An atomic transaction happens as one unit, either the whole thing
        commits or none of it does.
        \item \code{CONSISTENCY}: A consistent transaction brings the DB from one valid state to
        another valid state with respect to any constraints.
        \item \code{ISOLATION}: Concurrent isolated transactions would have the same result if run sequentially.
        \item \code{DURABILITY}: A committed transaction will remain committed even in the event of
        a hardware failure.
    \end{itemize}
\end{itemize}
    \end{multicols}

\end{document}